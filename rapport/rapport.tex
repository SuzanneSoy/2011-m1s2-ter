\documentclass[a4paper,11pt,french]{article}

\usepackage[frenchb]{babel}
\usepackage[T1]{fontenc}
\usepackage[utf8]{inputenc}
\usepackage{um2/um2}

\title{Rapport de TER\\---\\Reconception du jeu Pticlic sous Android}
\author{Yoann \textsc{Bonavero} \and Bertrand \textsc{Brun} \and John \textsc{Charron} \and Georges \textsc{Dupéron}}

\begin{document}

\maketitle

\tableofcontents
\newpage

\section{Difficultés rencontrées}
\subsection{Itération 1, semaine 1}
\begin{itemize}
\item Outil de création de diagrammes de GANTT (planner) est assez mauvais.
\item Lenteur de l'émulateur Android : impossible de travailler sur mon PC.% gd
\item Caractères non échappés dans le dump de la base.% gd
\end{itemize}

\section{Compte rendu des réunions}
\subsection{14 janvier 2010}
Outils :
\begin{itemize}
\item Langage Java
\item Eclipse (version la plus récente)
\item Plugin ADT (Android Developper Toolkit)
\item SDK Android
\item Github
\item LaTeX
\item JUnit peut-être
\end{itemize}

Durée du projet 4 mois (4 itérations de 4 semaines)

Conventions de code : \url{http://java.sun.com/docs/codeconv/html/CodeConventions.doc6.html}

Code (noms de variables, etc.) en anglais, commentaires en français, javadoc en français.

\subsection{26 janvier 2011}
Mettre le serveur (PHP) sur free.fr, pour pouvoir tester facilement

Utilisation d'une classe \verb!Constant!

Écran d'accueil du jeu : Image (splash), puis directement les icônes des modes de jeu + configuration, au lieu d'avoir un écran avec le logo et jouer/config, suivi du choix du mode de jeu.

\section{Deuxième section}
\section{Troisième section}
\newpage
\appendix
\section{Annexe A}
\section{Annexe B}

\end{document}
